\defmodule {num}

This module offers some useful constants and basic tools to 
manipulate numbers represented in different forms.

\bigskip\hrule
\code\hide
/* num.h for ANSI C */

#ifndef NUM_H
#define NUM_H
\endhide
#include "TestU01/gdef.h"
\endcode

%%%%%%%%%%%%%%%%%%%%%%%%%%%%%%%%%%%%%%%%%%
\guisec{Constants}
\code

#define num_Pi     3.14159265358979323846
\endcode
  \tab The number $\pi$.
  \endtab
\code

#define num_ebase  2.7182818284590452354
\endcode
  \tab The number $e$.
  \endtab
\code

#define num_Rac2   1.41421356237309504880
\endcode
  \tab $\sqrt{2}$, the square root of 2.
  \endtab
\code

#define num_1Rac2  0.70710678118654752440
\endcode
  \tab $1/\sqrt{2}$.
  \endtab
\code

#define num_Ln2    0.69314718055994530941
\endcode
  \tab $\ln(2)$, the natural logarithm of 2.
  \endtab
\code

#define num_1Ln2   1.44269504088896340737
\endcode
  \tab $1 / \ln(2)$.
  \endtab
\code

#define num_MaxIntDouble   9007199254740992.0
\endcode
  \tab Largest integer $n_0 = 2^{53}$ such that all integers
  $n \le n_0$ are represented  exactly as a {\tt double}.
  \endtab


%%%%%%%%%%%%%%%%%%%%%%%%%%%%%%%%%%%%%%%%%%
\guisec{Precomputed powers}
\code

#define num_MaxTwoExp   64
\endcode
  \tab Powers of 2 up to {\tt num\_MaxTwoExp} are stored exactly 
  in the array {\tt num\_TwoExp}.
  \endtab
\code

extern double num_TwoExp[];   
\endcode
  \tab  Contains precomputed powers of 2.
  One has {\tt num\_TwoExp[i]} $= 2^i$ for $0 \le i \le$ 
  {\tt num\_MaxTwoExp}.
\endtab
\code

#define num_MAXTENNEGPOW   16
\endcode
  \tab Negative powers of 10 up to {\tt num\_MAXTENNEGPOW} are stored
  in the array {\tt num\_TENNEGPOW}.
  \endtab
\code

extern double num_TENNEGPOW[];
\endcode
 \tab Contains the precomputed negative powers of 10.
   One has {\tt TENNEGPOW[j]}$ = 10^{-j}$, for $j=0,\ldots,$
 {\tt num\_MAXTENNEGPOW}.
\endtab


%%%%%%%%%%%%%%%%%%%%%%%%%%%%%%%%%%%%%%%%%%
\guisec{Prototypes}
\code

#define num_Log2(x) (num_1Ln2 * log(x)) 
\endcode
  \tab Gives the logarithm of $x$ in base 2.
  \endtab
\code


long num_RoundL (double x);
\endcode
  \tab Rounds $x$ to the nearest ({\tt long}) integer and returns it.
  \endtab
\code


double num_RoundD (double x);
\endcode
  \tab Rounds $x$ to the nearest ({\tt double}) integer and returns it.
  \endtab
\code


int num_IsNumber (char S[]);
\endcode 
\tab  Returns {\tt 1} if the string {\tt S} begins with a number     
   (with the possibility of spaces and a $+/-$ sign    
   before the number). For example, `` + 2'' and ``4hello''
   return {\tt 1}, while ``$-+2$'' and ``hello'' return
   {\tt 0}.
\endtab
\code


void num_IntToStrBase (long k, long b, char S[]);
\endcode
  \tab Returns in {\tt S} the string representation of {\tt k} in base
  {\tt b}.
  \endtab
\code


void num_Uint2Uchar (unsigned char output[], unsigned int input[], int L);
\endcode
  \tab Transforms the $L$ 32-bit integers contained in {\tt input} into
   $4L$ characters and puts them into {\tt output}. The order is such that 
  the 8 most significant bits of {\tt input[0]} will be in {\tt output[0]},
  the 8 least significant bits of {\tt input[0]} will be in {\tt output[3]},
  and the 8 least significant bits of {\tt input[L-1]} will be in
   {\tt output[4L-1]}. Array {\tt output} must have at least $4L$ elements.
  \endtab
\code


void num_WriteD (double x, int i, int j, int k);
\endcode
  \tab  Writes {\tt x} to current output.  Uses a total of at least {\tt i}
   positions (including the sign and point when they appear),
   {\tt j} digits after the decimal point and at least {\tt k}
   significant digits.   The number is rounded if necessary.
   If there is not enough space to print the number in decimal notation
   with at least {\tt k} significant digits
   ({\tt j} or {\tt i} is too small), it will be printed in scientific
   notation with at least {\tt k} significant digits.
   In that case, {\tt i} is increased if necessary. 
   Restriction: {\tt j} and {\tt k} must be strictly smaller than {\tt i}.
 \endtab
\code


void num_WriteBits (unsigned long x, int k);
\endcode
 \tab Writes {\tt x} in base 2 in a field of at least
  $\max\{b, |k|\}$ positions, where $b$ is the number of bits in an
  {\tt unsigned long}.
  If $k>0$, the number will be right-justified, otherwise left-justified.
 \endtab
\code


long num_MultModL (long a, long s, long c, long m);
\endcode
\tab  Returns $(as + c) \mod m$.  Uses the decomposition technique
  of \cite{rLEC91a} to avoid overflow. Supposes that $s < m$.
\endtab
\code


double num_MultModD (double a, double s, double c, double m);
\endcode
 \tab  Returns $(as+c) \mod m$, assuming that
  $a$, $s$, $c$, and $m$ are all {\em integers\/} less than $2^{35}$ 
  (represented exactly).
  Works under the assumption that all positive integers less than 
  $2^{53}$ are represented exactly in floating-point (in {\tt double}).
\endtab
\code


long num_InvEuclid (long m, long z);
\endcode
 \tab  This function computes the inverse $z^{-1}\bmod m$ by the 
  modified Euclid algorithm (see \cite[p. 325]{iKNU81a}) and returns
  the result. If the inverse does not exist, returns 0.
\endtab
\code


unsigned long num_InvExpon (int E, unsigned long z);
\endcode
 \tab
  This function computes the inverse  $z^{-1} \bmod 2^E$
  by exponentiation  and returns the result. If the inverse does not
  exist, returns 0.
  Restriction: $E$ not larger than the number of bits
  in an {\tt unsigned long}.
 \endtab

\code
\hide
#endif
\endhide
\endcode
